\documentclass{article}

%other packages
\usepackage[a4paper]{geometry}
\usepackage{longtable}
\usepackage{wrapfig}
\setlength\parindent{0pt}
\usepackage{enumitem}
\usepackage[table]{xcolor}
\usepackage{polynom}
\def\scaleint#1{\vcenter{\hbox{\scaleto[3ex]{\displaystyle\int}{#1}}}}
\usepackage{array}
\newcolumntype{C}{>{{}}c<{{}}} % for '+' and '-' symbols
\newcolumntype{R}{>{\displaystyle}r} % automatic display-style math mode 
\usepackage{tabularray}
\usepackage{dcolumn,tabularx,booktabs}

%maths
\usepackage{mathtools}
\usepackage{amsmath}
\usepackage{amssymb}
\usepackage{amsfonts}
\usepackage{autobreak}

%tikzpicture
\usepackage{tikz}
\usepackage{scalerel}
\usepackage{pict2e}
\usepackage{tkz-euclide}
\usepackage{tikz-3dplot}
\usetikzlibrary{calc}
\usetikzlibrary{patterns,arrows.meta}
\usetikzlibrary{shadows}
\usetikzlibrary{external}
\usetikzlibrary{decorations.pathreplacing,angles,quotes}

%pgfplots
\usepackage{pgfplots}
\pgfplotsset{compat=1.18}
\usepgfplotslibrary{statistics}
\usepgfplotslibrary{fillbetween}

\pgfplotsset{
    standard/.style={
    axis line style = thick,
    trig format=rad,
    enlargelimits,
    axis x line=middle,
    axis y line=middle,
    enlarge x limits=0.15,
    enlarge y limits=0.15,
    every axis x label/.style={at={(current axis.right of origin)},anchor=north west},
    every axis y label/.style={at={(current axis.above origin)},anchor=south east}
    }
}

\begin{document}

Math 115 - Week 2, Class 4 - 10 Jan 2024
\hrule

\begin{center}
Recall:
$\left\{\begin{aligned}
y=a^x,\quad a>0\\
e:\ \lim_{h\to0}\frac{e^h-1}{h}=1\\
(a^x)^\prime=a^x\lim_{h\to0}\frac{a^h-1}{h}
\end{aligned}\right.$
\end{center}

Now we will define $(a^x)^{-1}$.

\[y=a^x\Longleftrightarrow x=\log_ay\]

For $x$ sufficiently large,

$\begin{aligned}
a>1\qquad a^x&>a^\textnormal{any}\\
\log_ax&<x^\textnormal{any positive}
\end{aligned}$

\vspace{10pt}

{\bf{}Properties of Logarithms}
\begin{enumerate}

\item $\log_a(xy)=\log_a(x)+\log_a(y)$

Proof:
\begin{align*}
\textnormal{Let }x&=a^\alpha\quad y=a^\beta\\
(xy)&=a^\alpha a^\beta=a^{\alpha+\beta}\\
\log_a(xy)&=\alpha+\beta=\log_ax+\log_ay
\end{align*}

\item$\log_ax^k=k\log_ax$

Proof:
\begin{align*}
\textnormal{Let }x&=a^\alpha\quad\alpha=\log_ax\\
x^k&=(a^\alpha)^k=a^{k\alpha}\\
\log_ax^k&=k\alpha=k\log_ax
\end{align*}

\item $\log_a\left(\frac{x}{y}\right)=\log_ax-\log_ay$

Proof omitted due to similarity to (1)

\item $\log_ax=\frac{\log_bx}{\log_ba}$

Proof:
\begin{align*}
\textnormal{Let }y&=\log_ax\\
a^y&=x\\
y\log_b(a)&=\log_b(x)\\
\therefore\log_ax=\frac{\log_bx}{\log_ba}
\end{align*}

\item $\log_a1=0$

Proof: \[a^0=1\therefore0=\log_a1\]

\item $\log_e(x)=\ln(x)$ and is called the "natural logarithm."

Note that by $e$, Euler's ("Oilers") Constant is meant.

\item $\log_{10}x=\lg(x)$ and is called the "decimal logarithm"
\end{enumerate}

Now, with all of these logarithm rules in our toolkit, we will once again take a crack at differentiating the exponential function - this time with much more success. In fact, due to the fact that  the exponential derivative is a scalar multiple of the exponential function, we can with very much ease discover the antiderivative of the exponential function.

\begin{align*}
(a^x)^\prime&=(e^{x\ln a})^\prime\\
&=e^{x\ln a}\cdot\ln a\\
&=a^x\cdot\ln a\\
\Rightarrow\int a^x\ dx&=\frac{1}{\ln a}\int a^x\cdot\ln a\ dx\\
&=\frac{a^x}{\ln a}+C
\end{align*}

And finally, we will use our previously conceived definition of Euler's Constant to derive two more definitions.

\vspace{10pt}

Recall: $e$ is the number s.t. $\displaystyle\lim_{h\to0}\frac{e^h-1}{h}=1$

\begin{center}
Let $e^h-1=t\rightarrow
\left\{\begin{pmatrix}
h\to0^+&\quad t\to0^+\\
h\to0^-&\quad t\to0^-
\end{pmatrix}\right.$
\end{center}

\begin{center}
$\boxed{\begin{gathered}
\lim_{t\to0}\frac{\ln(1+t)}{t}=1
\end{gathered}}$
\end{center}

\[\textnormal{If }\lim_{x\to\textnormal{ any}}f(x)=1\textnormal{, then}\lim_{x\to\textnormal{ any}}\frac{1}{f(x)}=1\]

\[\textnormal{So, }\lim_{t\to0}\frac{t}{\ln(1+t)}=1\]

\begin{center}
$\boxed{\begin{gathered}
\lim_{t\to0}\frac{1}{t}\ln(1+t)=1
\end{gathered}}$
\end{center}

\[\lim_{t\to0}\ln(1+t)^\frac{1}{t}=1=\ln(\lim_{t\to0}(1+t)^\frac{1}{t})\]

\begin{center}
$\boxed{\begin{gathered}
\lim_{t\to0}(1+t)^\frac{1}{t}=e
\end{gathered}}$
\end{center}

Let $t=1/x$

\begin{center}
$\boxed{\begin{gathered}
\lim_{x\to\infty}\left(1+\frac{1}{x}\right)^x=e
\end{gathered}}$
\end{center}








\end{document}