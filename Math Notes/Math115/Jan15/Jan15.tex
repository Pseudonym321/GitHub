\documentclass{article}

%other packages
\usepackage[a4paper]{geometry}
\usepackage{longtable}
\usepackage{wrapfig}
\setlength\parindent{0pt}
\usepackage{enumitem}
\usepackage[table]{xcolor}
\usepackage{polynom}
\def\scaleint#1{\vcenter{\hbox{\scaleto[3ex]{\displaystyle\int}{#1}}}}
\usepackage{array}
\newcolumntype{C}{>{{}}c<{{}}} % for '+' and '-' symbols
\newcolumntype{R}{>{\displaystyle}r} % automatic display-style math mode 
\usepackage{tabularray}
\usepackage{dcolumn,tabularx,booktabs}
\usepackage[most]{tcolorbox}

%maths
\usepackage{mathtools}
\usepackage{amsmath}
\usepackage{amssymb}
\usepackage{amsfonts}
\usepackage{autobreak}

%tikzpicture
\usepackage{tikz}
\usepackage{scalerel}
\usepackage{pict2e}
\usepackage{tkz-euclide}
\usepackage{tikz-3dplot}
\usetikzlibrary{calc}
\usetikzlibrary{patterns,arrows.meta}
\usetikzlibrary{shadows}
\usetikzlibrary{external}
\usetikzlibrary{decorations.pathreplacing,angles,quotes}

%pgfplots
\usepackage{pgfplots}
\pgfplotsset{compat=1.18}
\usepgfplotslibrary{statistics}
\usepgfplotslibrary{fillbetween}

\pgfplotsset{
    standard/.style={
    axis line style = thick,
    trig format=rad,
    enlargelimits,
    axis x line=middle,
    axis y line=middle,
    enlarge x limits=0.15,
    enlarge y limits=0.15,
    every axis x label/.style={at={(current axis.right of origin)},anchor=north west},
    every axis y label/.style={at={(current axis.above origin)},anchor=south east}
    }
}

\begin{document}

Math 115 - Week 3, Class 6 - 15 Jan 2024
\hrule

\vspace{10pt}

We started class with answering some of the Exponents and Logarithms Exercises from the Meskanas page.

\vspace{10pt}

{\bf{}EXAMPLE 10.} Given $5^2x-7^x-5^{2x}\cdot17+7^x\cdot17=0$, find $x$.

\begin{align*}
\frac{0}{7^x}&=\frac{5^x}{7^x}-1-17\left(\frac{25}{7}\right)^x+17\\
5^{2x}&=(5^2)^x=25^x\\
0&=\left(\frac{25}{7}\right)^x\cdot(-16)+16\\
16&=16\cdot\left({25}{7}\right)^x\\
1&=\left({25}{7}\right)^x\\
\therefore x&=0
\end{align*}

{\bf{}EXAMPLE 8.} Given $5\cdot4^{x-1}-16^x+(0.25)\cdot2^{x+2}+7=0$, find $x$.

\begin{align*}
5\cdot\frac{4^x}{4}-(4^x)^2+\frac{1}{4}\cdot(2^2)^x\cdot2^2+7&=0\\
\textnormal{Let }y&=4^x\\
\frac{5}{4}y-y^2+y+7&=0\\
y^2-\frac{9}{4}y-7&=0\\
y&=\frac{-b\pm\sqrt{b^2-4ac}}{2a}\\
&=\frac{\frac{9}{4}\pm\sqrt{\frac{81}{16}+28}}{2}\\
&=\frac{9}{8}\pm\frac{1}{2}\sqrt{\frac{529}{16}}\\
&=\frac{9}{8}\pm\frac{23}{8}\\
&=+4\quad (y=4^x>0)\\
\therefore x&=1
\end{align*}

{\bf{}EXAMPLE 6.} Evaluate $\displaystyle27^{-4\log_{1/3}5}$

\begin{align*}
27^{-4\log_{1/3}5}&=27^{-\log_{1/3}5^4}\\
&=\left(\left(\frac{1}{3}\right)^{-3}\right)^{-\log_{1/3}5^4}\\
&=\left(\frac{1}{3}\right)^{3\log_{1/3}5}\\
&=\left(\frac{1}{3}\right)^{\log_{1/3}5^{12}}\\
&=5^{12}
\end{align*}

{\bf{}EXAMPLE 8.} Evaluate $\displaystyle\log_{5/2}\frac{8}{125}$

\begin{align*}
\log_{5/2}\frac{8}{125}&=\log_{5/2}\left(\frac{125}{8}\right)^{-1}\\
&=-\log_{5/2}\left(\frac{5}{2}\right)^3\\
&=-3
\end{align*}

{\bf{}EXAMPLE 15.} Given $\displaystyle4^{|x+1|}=8$, solve for $x$.

\begin{align*}
4^{|x+1|}&=8\\
2^{2|x+1|}&=2^3\\
|x+1|&=\frac{3}{2}\\
x+1&=\pm\frac{3}{2}\\
x&=\pm\frac{3}{2}-1\\
x_1&=\frac{1}{2}\\
x_2&=-\frac{5}{2}
\end{align*}

{\bf{}EXAMPLE} Evaluate $\displaystyle\lim_{x\to\infty}e^{-2x}\cos x$

\vspace{10pt}

So, $e^{-2x}=(e^2)^{-x}$. Since $2<e<3$, $4<e^2<9$, therefore $e^2>1$. Since the base is greater than one and the exponent is a linear function with negative slope, we know that the shape of the graph is an exponentially decaying curve.

\begin{center}
\begin{tikzpicture}[]
\begin{axis}[
standard,
xmin=-2, xmax=2,
ymin=0, ymax=4,
xtick={\empty}, ytick={\empty}]
\addplot[samples=200,domain=-3:3] {e^(-x)};
\end{axis}
\end{tikzpicture}
\end{center}

Cosine is a bounded function, specifically $-1\leq\cos x\leq1$. We can multiply eacy term by $e^{-2x}$ to obtain an inequality where the center term is our original expression, and where each terminal term approaches the same value when the limit is applied. The inequality we obtain is $-e^{-2x}\leq e^{-2x}\cos x\leq e^{-2x}$.

\vspace{10pt}

\begin{tabularx}{\textwidth}{|c|}
\cline{1-1}\\
\multicolumn{1}{|X|}{{\bf{}The Squeeze Theorem} If $f(x)\leq g(x)\leq h(x)$ for all $x$ in an open interval that contains $a$ (except pollibly at $a$) and}\\[2em]
$\displaystyle\lim_{x\to a}f(x)=\lim_{x\to a}h(x)=L$\\[1.5em]
\multicolumn{1}{|l|}{then}\\[1em]
$\displaystyle\lim_{x\to a}g(x)=L$\\[1.5em]
\cline{1-1}
\end{tabularx}

\vspace{10pt}

Our Math Professor calls it "The Sandwich Theorem". Stewart calls it "The Squeeze Theorem". Some people call it "The Pinching Theorem". Others call it "The Police Theorem". It's at the point where we could joke about the number of names of this theorem approaching infinity.

\vspace{10pt}

Despite the silly sounding names, this theorem is actually very useful. All it says is that if two functions approach the same value at an input, then any function that is strictly between them (except possibly at that input, given the nature of limits) will approach the same value at that input.

\vspace{10pt}

By taking $x$ to be sufficiently big, we can make $e^{-2x}$ arbitrarily close to zero, regardless of the sign in front. So, since both terminal terms approach zero, we can use the "Theorem of Many Names" - as I've decided to call it for reasons of humor - to deduce that

\[\lim_{x\to\infty}e^{-2x}\cos x=0\]

\vspace{10pt}

{\bf{}EXAMPLE} Evaluate $\displaystyle\int_0^1\frac{\sqrt{1+e^{-x}}}{e^x}\ dx$

\vspace{10pt}

First we solve the indefinite integral, so that we don't have to keep manipulating the boundaries every time we make a substitution.

\begin{align*}
\int\frac{\sqrt{1+e^{-x}}}{e^x}\ dx&=\int\sqrt{1+e^{-x}}e^{-x}\ dx\\
&=\left(\begin{array}{cc}t=1+e^{-x}\\dt=-e^{-x}\ dx\\-dt=e^{-x}\ dx\end{array}\right)\\
&=\int\sqrt{t}\ (-dt)\\
&=-\frac{t^{3/2}}{3/2}\\
&=-\frac{2}{3}(1+e^{-x})^{3/2}
\end{align*}

Now having obtained the indefinite integral in terms of our orininal variable, we can just plug in the original limits of integration and evaluate using arithmetic.

\begin{align*}
\int_0^1\frac{\sqrt{1+e^{-x}}}{e^x}\ dx&=\left.-\frac{2}{3}(1+e^{-x})^{3/2}\right|_0^1\\
&=-\frac{2}{3}\left((1+e^{-1})^{3/2}-(1+e^0)^{3/2}\right)\\
&=-\frac{2}{3}\left(\left(1+\frac{1}{e}\right)^{3/2}-1\right)
\end{align*}

{\bf{}EXAMPLE} Differentiate $\displaystyle(x^2+1)^{\sin3x}$

\begin{align*}
y&=(x^2+1)^{\sin3x}\\
\ln y&=\sin3x\ln(x^2+1)\\
\frac{1}{y}y^\prime&=3\cos3x\ln(x^2+1)+\frac{2x\sin3x}{x^2+1}\\
\therefore y&=(x^2+1)^{\sin3x}\left[3\cos3x\ln(x^2+1)+\frac{2x\sin3x}{x^2+1}\right]
\end{align*}

\vspace{10pt}

\begin{center}
\begin{tabular}{|ll|}
\hline&\\
\multicolumn{2}{|l|}{{\bf{}Important Equations}}\\[1em]
$(a^x)^\prime=a^x\ln a$ & $(e^x)^\prime=e^x$\\[1em]
$(\ln x)^\prime=\frac{1}{x}$ & $\displaystyle\int a^x\ dx=\frac{a^x}{\ln a}$\\[1.5em]
\multicolumn{2}{|c|}{$\displaystyle\int\frac{1}{x}\ dx=\ln|x|+C$}\\[1.5em]
\hline
\end{tabular}
\end{center}

\end{document}