\documentclass{article}

%other packages
\usepackage[a4paper]{geometry}
\usepackage{longtable}
\usepackage{wrapfig}
\setlength\parindent{0pt}
\usepackage{enumitem}
\usepackage[table,dvipsnames]{xcolor}
\usepackage{polynom}
\def\scaleint#1{\vcenter{\hbox{\scaleto[3ex]{\displaystyle\int}{#1}}}}
\usepackage{array}
\newcolumntype{C}{>{{}}c<{{}}} % for '+' and '-' symbols
\newcolumntype{R}{>{\displaystyle}r} % automatic display-style math mode 
\usepackage{tabularray}
\usepackage{dcolumn,tabularx,booktabs}
\usepackage[most]{tcolorbox}
%\graphicspath{ {C:/Users/twill/OneDrive/Desktop/Eliason/Diagrams} }

%maths
\usepackage{mathtools}
\usepackage{amsmath}
\usepackage{amssymb}
\usepackage{amsfonts}
\usepackage{autobreak}

%tikzpicture
\usepackage{tikz}
\usepackage{scalerel}
\usepackage{pict2e}
\usepackage{tkz-euclide}
\usepackage{tikz-3dplot}
\usetikzlibrary{calc}
\usetikzlibrary{patterns,arrows.meta}
\usetikzlibrary{shadows}
\usetikzlibrary{external}
\usetikzlibrary{decorations.pathreplacing,angles,quotes}

%pgfplots
\usepackage{pgfplots}
\pgfplotsset{compat=1.18}
\usepgfplotslibrary{statistics}
\usepgfplotslibrary{fillbetween}

\pgfplotsset{
    standard/.style={
    axis line style = thick,
    trig format=deg,
    enlargelimits,
    axis x line=middle,
    axis y line=middle,
    enlarge x limits=0.15,
    enlarge y limits=0.15,
    every axis x label/.style={at={(current axis.right of origin)},anchor=north west},
    every axis y label/.style={at={(current axis.above origin)},anchor=south east}
    }
}

\begin{document}

Math 115, 15 March 2024
\hrule

\vspace{10pt}

Learning Outcome: determine if $\int$ is convergent and evaluate it if it is.

\vspace{10pt}

If a function $f(x)$ is less than or equal to another function $g(x)$ for all $x\in\mathbb{R}$ such that $x\geq M$, then the following is true;

\begin{center}
\begin{enumerate}[label=(\alph*)]
\item $\displaystyle\int_M^\infty f(x)\ dx\mbox{ convg.}\implies\int_M^\infty g(x)\ dx\mbox{ convg.}$
\item $\displaystyle\int_M^\infty g(x)\ dx\mbox{ diverg.}\implies\int_M^\infty g(x)\ dx\mbox{ diverg.}$
\end{enumerate}
\end{center}

\vspace{10pt}

{\bf{}EXAMPLE} $\displaystyle\int_{10}^\infty\frac{1}{x\ln^2x}\ dx$

\begin{align*}
\int_{10}^\infty\frac{1}{x\ln^2x}\ dx&=(\ln x=u,\ du=1/x\ dx)\\
&=\int_{\ln10}^\infty\frac{1}{u^2}\ du\mbox{ convg. by p-test}
\end{align*}

\vspace{10pt}

{\bf{}EXAMPLE} $\displaystyle\int_{10}^\infty\frac{1+\sin^23x+\arctan x}{x\ln^2x}\ dx$

\[\mbox{numerator }<1+1+\pi/2<4\]

So,

\[\int_{10}^\infty\frac{1+\sin^23x+\arctan x}{x\ln^2x}\ dx<\int_{10}^\infty\frac{4}{x\ln^2x}\ dx\mbox{ convg. by p-test}\]

Therefore, our original integral is convergent by comparison.

\vspace{10pt}

O, o - notation:

\[f(x)=O(g(x))\iff\exists c:f(x)\leq cg(x)\]

A function is a big O of another function if it is strictly less than or equal to a scalar multiple of another function.

\[f(x)=O(g(x))\land\int_a^\infty g(x)\ dx\mbox{ convg.}\implies\int_a^\infty f(x)\ dx\mbox{ convg.}\]

If a function is big O of another function and that function is convergent on the domain, then our original function is also. Similarly;

\[f(x)=O(g(x))\land\int_a^\infty f(x)\ dx\mbox{ diverg.}\implies\int_a^\infty g(x)\ dx\mbox{ diverg.}\]

\newpage

$f(x)\sim g(x)$ when $x\to\infty$ if $f(x)=O(g(x)$ and $g(x)=O(f(x))$

\vspace{10pt}

{\bf{}EXAMPLE} $\displaystyle5g(x)<f(x)<9g(x)$

\vspace{10pt}

If $f(x)\sim g(x)$, then they converg or diverge simultaneously. This implies

\[\lim_{x\to\infty}\frac{f(x)}{g(x)}=c\]

and

\[\left|\frac{f(x)}{g(x)}-c\right|<\varepsilon\implies(c-\varepsilon)g(x)<f(x)<(c+\varepsilon)g(x)\]

or, "$f(x)\sim g(x)$ when $x\to\infty$."

\vspace{10pt}

{\bf{}EXAMPLE} $I_1=\int_1^\infty\frac{2+e^{-x}}{x}\ dx>I_2=\int_1^\infty\frac{2}{x}\ dx\mbox{ diverg.}\therefore I_1\mbox{ diverg. too}$

\vspace{10pt}

{\bf{}EXAMPLE} $I_1=\int_1^\infty\frac{2-e^{-x}}{x}\ dx$

\[\lim_{x\to]infty}\left(\frac{2-e^{-x}}{x}\div\frac{2}{x}\right)=\lim_{x\to\infty}\frac{2-e^{-x}}{2}=1\]

so,

\[\int_1^\infty\frac{2-e^{-x}}{x}\ dx\mbox{ and }\int_1^\infty\frac{2}{x}\ dx\mbox{ are similar, i.e. converg or diverge simultaneously}\]

\vspace{10pt}

{\bf{}EXAMPLE} $\int_0^\infty\frac{\arctan x}{2+e^x}\ dx$

\[\frac{-\pi/2}{2+e^x}<\frac{\arctan x}{e^x+2}<\frac{\pi/2}{2+e^x}<\frac{\pi}{2+e^x}\]

\[\implies\int_0^\infty\frac{\pi}{2}\frac{1}{e^x}\ dx=\lim_{b\to\infty}\frac{\pi}{2}(-e^{-x})\big|_0^b=\frac{\pi}{2}\therefore\mbox{ converg.}\]

\vspace{10pt}

{\bf{}MODIFICATION} $\int_0^\infty\frac{\arctan x}{e^x-x^2-3}\ dx$

\vspace{10pt}

While it is not generally true that

\[\frac{\arctan x}{e^x-x^2-3}<\frac{\pi}{2e^x}\]

but it is the truth that

\[\lim_{x\to\infty}\frac{\arctan x}{e^x-x^2-3}\div\frac{1}{e^x}=\frac{\pi}{2}\]

since the terms are equivalent, or "similar"

\newpage

{\bf{}EXAMPLE} $\int_1^\infty\frac{x\sqrt{x}-2x+1}{x^3+2x^2+x+5}\ dx$

\[\mbox{inegrand }<\frac{x\sqrt{x}}{x^3}=x^{-3/2}\]

\[\int x^{-3/2}\ dx\mbox{ convg.}\therefore\int\mbox{ convg.}\]

\vspace{10pt}

{\bf{}MODIFICATION} $\int_{10}^\infty\frac{\arctan x}{x^3-2x^2-5x-6}\ dx$

\[\lim_{x\to\infty}\mbox{ integrand}\div\mbox{ dominating terms}=1\]

\vspace{10pt}

Inproper $\int$ with a singularity (discontinuity);

\[f(x):x\in[a,b)\cup(b,c]\to\mathbb{R}\implies\int_a^cf(x)\ dx=\lim_{n\to b^-}\int_a^nf(x)+\lim_{n\to b^+}\int_n^cf(x)\ dx\]



\end{document}
