\documentclass{article}

%other packages
\usepackage[a4paper]{geometry}
\usepackage{longtable}
\usepackage{wrapfig}
\setlength\parindent{0pt}
\usepackage{enumitem}
\usepackage[table,dvipsnames]{xcolor}
\usepackage{polynom}
\def\scaleint#1{\vcenter{\hbox{\scaleto[3ex]{\displaystyle\int}{#1}}}}
\usepackage{array}
\newcolumntype{C}{>{{}}c<{{}}} % for '+' and '-' symbols
\newcolumntype{R}{>{\displaystyle}r} % automatic display-style math mode 
\usepackage{tabularray}
\usepackage{dcolumn,tabularx,booktabs}
\usepackage[most]{tcolorbox}
%\graphicspath{ {C:/Users/twill/OneDrive/Desktop/Eliason/Diagrams} }

%maths
\usepackage{mathtools}
\usepackage{amsmath}
\usepackage{amssymb}
\usepackage{amsfonts}
\usepackage{autobreak}

%tikzpicture
\usepackage{tikz}
\usepackage{scalerel}
\usepackage{pict2e}
\usepackage{tkz-euclide}
\usepackage{tikz-3dplot}
\usetikzlibrary{calc}
\usetikzlibrary{patterns,arrows.meta}
\usetikzlibrary{shadows}
\usetikzlibrary{external}
\usetikzlibrary{decorations.pathreplacing,angles,quotes}

%pgfplots
\usepackage{pgfplots}
\pgfplotsset{compat=1.18}
\usepgfplotslibrary{statistics}
\usepgfplotslibrary{fillbetween}

\pgfplotsset{
    standard/.style={
    axis line style = thick,
    trig format=deg,
    enlargelimits,
    axis x line=middle,
    axis y line=middle,
    enlarge x limits=0.15,
    enlarge y limits=0.15,
    every axis x label/.style={at={(current axis.right of origin)},anchor=north west},
    every axis y label/.style={at={(current axis.above origin)},anchor=south east}
    }
}

\begin{document}

\begin{enumerate}

\item[1.] One of the legs of a right triangle has length 4 cm. Express the length of the altitude perpendicular to the hypotenuse as a function of the length of the hypotenuse.

\begin{center}
\begin{tikzpicture}
\coordinate (O) at  (0,0);
\coordinate (E) at  (4,0);
\coordinate (N) at  (0,5);
\coordinate (P) at  (100/41,{(-5/4)*(100/41)+5});

\draw[] (O) -- node[pos=0.5,below]{4} (E) -- node[pos=0.5,above right]{$x$} (P) -- node[pos=0.5,above right]{$h-x$} (N) -- node[pos=0.5,left]{$y$} cycle;
\draw[] (O) -- node[pos=0.5,above]{$a$} (P);
\draw pic[draw,-,angle eccentricity=1.4, angle radius=0.3cm]{right angle=E--P--O};
\draw pic[draw,-,angle eccentricity=1.4, angle radius=0.3cm]{right angle=E--O--N};

\draw[decoration={brace,raise=50pt},decorate]
  (N) -- node[pos=0.5,above right=52pt]{$h$} (E);
\end{tikzpicture}
\end{center}

By using the area formula for a triangle, $\frac{1}{2}(base)(height)$, in two ways, we see that $\frac{1}{2}(4)(y)=\frac{1}{2}(h)(a)$, so $a=\frac{4y}{h}$. Since $4^2+y^2=h^2$, $y=\sqrt{h^2-16}$, and $a=\frac{4\sqrt{h^2-16}}{h}$.

\item[2.]





\end{enumerate}




\end{document}
