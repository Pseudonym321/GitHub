\documentclass{article}

%other packages
\usepackage[a4paper]{geometry}
\usepackage{longtable}
\usepackage{wrapfig}
\setlength\parindent{0pt}
\usepackage{enumitem}
\usepackage[table]{xcolor}
\usepackage{polynom}
\def\scaleint#1{\vcenter{\hbox{\scaleto[3ex]{\displaystyle\int}{#1}}}}
\usepackage{array}
\newcolumntype{C}{>{{}}c<{{}}} % for '+' and '-' symbols
\newcolumntype{R}{>{\displaystyle}r} % automatic display-style math mode 
\usepackage{tabularray}
\usepackage{dcolumn,tabularx,booktabs}
\usepackage{esvect}
\usepackage{comment}

%maths
\usepackage{mathtools}
\usepackage{amsmath}
\usepackage{amssymb}
\usepackage{amsfonts}
\usepackage{autobreak}

%tikzpicture
\usepackage{tikz}
\usepackage{scalerel}
\usepackage{pict2e}
\usepackage{tkz-euclide}
\usepackage{tikz-3dplot}
\usetikzlibrary{calc}
\usetikzlibrary{patterns,arrows.meta}
\usetikzlibrary{shadows}
\usetikzlibrary{external}
\usetikzlibrary{decorations.pathreplacing,angles,quotes}
\usetikzlibrary{perspective,spath3}

%pgfplots
\usepackage{pgfplots}
\pgfplotsset{compat=1.18}
\usepgfplotslibrary{statistics}
\usepgfplotslibrary{fillbetween}

\pgfplotsset{
    standard/.style={
    axis line style = thick,
    trig format=rad,
    enlargelimits,
    axis x line=middle,
    axis y line=middle,
    enlarge x limits=0.15,
    enlarge y limits=0.15,
    every axis x label/.style={at={(current axis.right of origin)},anchor=north west},
    every axis y label/.style={at={(current axis.above origin)},anchor=south east}
    }
}

\begin{document}

Math 115 - 12 Feb 2024
\hrule

\vspace{10pt}

We started class with an integration by parts after u-substitution example. It makes useof a special property of cubic variables in integration. That being said, we could also have just done it through multiple iterations of integration by parts. Often there are many ways to solve one integral. Personally, I prefer to "take a sledge hammer to a nut" and use the techniques which are broader in scope. That being said, it is essential for aspiring mathematicians to not neglect these "little tricks," because they are standard techniques.

\begin{align*}
\int\theta^3\cos(\theta^2)\ d\theta&\left(\begin{aligned}\theta^3=\theta\cdot\theta^2\\\theta\ d\theta=\frac{1}{2}\ d(\theta^2)\end{aligned}\right)\\
&=\int\theta^2\cos(\theta^2)\theta\ d\theta\\
&=\left(\begin{aligned}\theta^2=t\\\theta\ d\theta=\frac{1}{2}\ dt\end{aligned}\right)\\
&=\frac{1}{2}\int t\cos t\ dt\\
&=\left(\begin{aligned}u=t&\quad du=\ dt\\\cos t\ dt=\ dv&\quad v=\sin t\end{aligned}\right)\\
&=\frac{1}{2}t\sin t-\frac{1}{2}\int\sin t\ dt\\
&=\frac{1}{2}(t\sin t+\cos t)+C
\end{align*}

\vspace{10pt}

We are also studying integrals of rational functions of polynomials - using partial fraction decomposition. This technique seems hard at first, but it is actually really easy! Let me explain.

\vspace{10pt}

Essentially, you just factor the denominator into its linear and irreducible quadratic factors, and then use a simple pattern to rewrite the quotient as a sum of simpler fractions. Where it can get a little confusing is where you have factors which are raised to degreeswhich are greater than one. In these cases, you just add another fraction for each power. That is, if you have, say, a proper rational integral of the form $\int\frac{\mbox{something}}{(\mbox{a linear function})^5}\ dx$, you'd just write a fraction for each integer power of that linear function, ranging from one to five. If it is improper (meaning that the degree of the numerator is greater than that of the denominator, you need to use a technique which I will outline in the next paragraph).

\newpage

For improper rational integrands, things are more complicated because you need to determine at least one zero of the polynomial numerator. Luckily for us, the questions will probably be nice and have integer (or obvious by some other means) roots. At the higher levels though, this finding of roots - especially of multivariate polynomials - is actually \textit{quite} advanced, and falls within the realm of algebraic geometry.

\vspace{10pt}

If you are having trouble with the factor theorem, I would strongly suggest you take a look at stewart's review of algebra. I love this resource and credit it for my calculus success. Don't be mislead by the title containing the word "basic," this stuff is \textit{really} important - and I spent months studying it before coming back to school after taking a year off. Yes, this 12 page document took me several months to thoroughly grasp, and now things are a \textit{lot} easier for me. This stuff is important.

\vspace{10pt}

After identifying a zero with the factor theorem, you can use polynomial long division to simplify the integrand. A useful trick for this actually is to guess factors based on the factors of the denominator ;). Here's a polynomial long division example that I copied from stewart's review of algebra.

\[\setlength\arraycolsep{0pt} 
\setlength\extrarowheight{2pt}
\begin{array}[t]{ RCCRCRCRCR }
	& & x^2 & - & x & - & 6 & & &\\
\cline{2-9}
	x-2 & \kern-0.4pt\raisebox{1.6pt}{\big)} & x^3 & - & 3x^2 & - & 4x & + & 12 \\
	& & x^3 & - & 2x^2 & & & & & \\
\cline{3-7}
	& & & - & x^2 & - & 4x & & & \\
	& & & - & x^2 & + & 2x & & & \\
\cline{4-9}
	& & & & & - & 6x & + & 12 & \\
	& & & & & - & 6x & + & 12 & \\
\cline{6-10}
	& & & & & & & & 0 &
\end{array}\]

\vspace{10pt}

This allows us to re-espress improper rational functions as the sum of a polynomial and a proper rational function. That is, where the degree of the numerator is greater than {\bf{}or equal to} the degree of the denominator;

\[\mbox{Improper Rational}(x)=\mbox{Polynomial}+\mbox{Proper Rational}\]

\newpage

There are four scenarios which we might encounter when doing partial fraction decomposition. And they are not mutually exclusive; a function may require more than one technique me used. Using one of them does not make it impossible to use another. You'll learn this by doing a few complicated examples after learning the four cases individually.

\vspace{10pt}

Case 1 deals with what we've already seen studying integrals of reciprocal functions of linears.

\begin{equation}
\int\frac{A}{x-a}\ dx=A\ln|x-a|+C
\end{equation}

\vspace{10pt}

Case 2 is for linear denominators which are raised to a power that is greater than one, meaning that we can just use the inverse power rule.

\begin{equation}
\int\frac{A}{(x-a)^k}\ dx=A\int(x-a)^{-k}\ d(x-a)=A\frac{(x-a)^{-k+1}}{-k+1}+C
\end{equation}

\newpage

Case 3 gets  a  bit more involved, as it deals with irreducible quadratic denominators (without repetition). Essentially, we complete the square in the denominator and simplify which gives us the sum of two integrals. One will be a really easy u-substitution one, and the other will require the inverse derivative of arctangent. ie. $\int1/(x^2+a^2)\ dx=(1/a)\arctan(x/a)+C$

\begin{align*}
\int\frac{ux+N}{x^2+px+q}\ dx&=\int\frac{ux+N}{\left(x+\frac{p}{2}\right)^2+\left(\frac{4\sqrt{q}-p}{2}\right)^2}\ dx\\
&=\left(\begin{aligned}x+\frac{p}{2}=t\\dx=\ dt\\x=t-\frac{p}{2}\end{aligned}\right)\\
&=\int\frac{M\left(t-\frac{p}{2}\right)+N}{t^2+\left(\frac{4\sqrt{q}-p}{2}\right)^2}\\
&=\int\frac{Mt+\left(N-\frac{Mp}{2}\right)}{t^2+\left(\frac{4\sqrt{q}-p}{2}\right)^2}\ dt\\
&=M\int\frac{t}{t^2+\left(\frac{4\sqrt{q}-p}{2}\right)^2}\ dt+\left(N-\frac{Mp}{2}\right)\int\frac{1}{t^2+\left(\frac{4\sqrt{q}-p}{2}\right)^2}\ dt\\
&=\left(\begin{aligned}t\ dt=\frac{1}{2}\ d\left(t^2+\left(\frac{4\sqrt{q}-p}{2}\right)^2\right)\\t^2+\left(\frac{4\sqrt{q}-p}{2}\right)^2=u\end{aligned}\right)\\
&=\frac{M}{2}\int\frac{du}{u}+\left(N-\frac{Mp}{2}\right)\cdot\frac{1}{\left(\frac{4\sqrt{q}-p}{2}\right)}\arctan\frac{t}{\left(\frac{4\sqrt{q}-p}{2}\right)}\\
&=\frac{M}{2}\ln(t^2+a^2)+\left(N-\frac{Mp}{2}\right)\cdot\frac{1}{\left(\frac{4\sqrt{q}-p}{2}\right)}\arctan\frac{t}{\left(\frac{4\sqrt{q}-p}{2}\right)}+C\\

&=\frac{M}{2}\ln(x^2+px+q)+\left(N-\frac{Mp}{2}\right)\frac{1}{\sqrt{q-\frac{p^2}{4}}}\cdot\arctan\frac{x+p/2}{\sqrt{1-\frac{p^2}{4}}}+C
\end{align*}

\newpage

{\bf{}EXAMPLE} Evaluate $\displaystyle\int\frac{3x+4}{x^2-3x+7}\ dx$

\vspace{10pt}

\begin{align*}
\int\frac{3x+4}{x^2-3x+7}\ dx&=\left(\begin{aligned}x^2-3x+7=\left(x-\frac{3}{2}\right)^2+\frac{19}{4}\\x-\frac{3}{2}=t\qquad x=t+\frac{3}{2}\\dx=\ dt\end{aligned}\right)\\
&=\int\frac{3(t+3/2)+4}{t^2+19/4}\ dt\\
&=3\int\frac{t}{t^2+19/2}\ dt+\frac{17}{2}\int\frac{1}{t^2+19/4}\ dt\\
&=\frac{3}{2}\int\frac{1}{t^2+19/4}\ d(t^2+19/4)+\frac{17}{2}\cdot\frac{2}{\sqrt{19}}\arctan\left(\frac{2t}{\sqrt{19}}\right)\\
&=\frac{3}{2}\ln(t^2+19/4)+\frac{17}{\sqrt{19}}\arctan\left(\frac{2t}{\sqrt{19}}\right)+C\\
&=\frac{3}{2}\ln(x^2-3x+7)+\frac{17}{\sqrt{19}}\arctan\left(\frac{2x-2}{\sqrt{19}}\right)+C
\end{align*}

\newpage

We then looked at the fourth type of partial fraction decomposition, which deals with repeated irreducible quadratics in the denominator. If you know case 3, then this case is very easy to learn. After completing the square, we get an integral of the form

\[\int\frac{At+B}{(t^2+a^2)^k}\ dt\]

\vspace{10pt}

After splitting it up, the one with $t$ in the numerator becomes a simple u-substitution one, and so we will focus on the other one in the next class (as class finished at this point).








\end{document}