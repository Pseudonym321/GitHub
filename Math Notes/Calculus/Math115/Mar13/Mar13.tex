\documentclass{article}

%other packages
\usepackage[a4paper]{geometry}
\usepackage{longtable}
\usepackage{wrapfig}
\setlength\parindent{0pt}
\usepackage{enumitem}
\usepackage[table,dvipsnames]{xcolor}
\usepackage{polynom}
\def\scaleint#1{\vcenter{\hbox{\scaleto[3ex]{\displaystyle\int}{#1}}}}
\usepackage{array}
\newcolumntype{C}{>{{}}c<{{}}} % for '+' and '-' symbols
\newcolumntype{R}{>{\displaystyle}r} % automatic display-style math mode 
\usepackage{tabularray}
\usepackage{dcolumn,tabularx,booktabs}
\usepackage[most]{tcolorbox}

%maths
\usepackage{mathtools}
\usepackage{amsmath}
\usepackage{amssymb}
\usepackage{amsfonts}
\usepackage{autobreak}

%tikzpicture
\usepackage{tikz}
\usepackage{scalerel}
\usepackage{pict2e}
\usepackage{tkz-euclide}
\usepackage{tikz-3dplot}
\usetikzlibrary{calc}
\usetikzlibrary{patterns,arrows.meta}
\usetikzlibrary{shadows}
\usetikzlibrary{external}
\usetikzlibrary{decorations.pathreplacing,angles,quotes}

%pgfplots
\usepackage{pgfplots}
\pgfplotsset{compat=1.18}
\usepgfplotslibrary{statistics}
\usepgfplotslibrary{fillbetween}

\pgfplotsset{
    standard/.style={
    axis line style = thick,
    trig format=deg,
    enlargelimits,
    axis x line=middle,
    axis y line=middle,
    enlarge x limits=0.15,
    enlarge y limits=0.15,
    every axis x label/.style={at={(current axis.right of origin)},anchor=north west},
    every axis y label/.style={at={(current axis.above origin)},anchor=south east}
    }
}

\begin{document}

Math 115 - 13 March 2024
\hrule

\vspace{10pt}

We discussed improper integrals, which are integrals with unbounded limits of integration.

\begin{align}
\int_a^\infty f(x)\ dx&=\lim_{b\to\infty}\int_a^bf(x)\ dx\\
\int_{-\infty}^a f(x)\ dx&=\lim_{c\to-\infty}\int_c^af(x)\ dx\\
\int_{-\infty}^\infty f(x)\ dx&=\lim_{b\to\infty}\lim_{c\to-\infty}\int_c^bf(x)\ dx
\end{align}

\vspace{10pt}

If the limit exists, then the integral converges; if it does not, then the integral is divergent.

\vspace{10pt}

Integrals of the form

\[\int_1^\infty\frac{1}{x^P}\ dx\]

are convergent if $P>1$, and divergent for $P\leq1$. This integral is often used for comparison tests.

\vspace{10pt}

{\bf{}EXAMPLE} $\displaystyle\int_1^\infty\frac{3+\sin2x+e^{-x}}{x^2+1}\ dx$

\[3+\sin2x+e^{-x}<5\Rightarrow\frac{3+\sin2x+e^{-x}}{x^2+1}<\frac{5}{x^2+1}<\frac{5}{x^2}\]

Compare to $\displaystyle\int_1^\infty\frac{5}{x^2}\ dx=-\frac{5}{x}\big|_1^\infty\Rightarrow\mbox{ converges.}$

\vspace{10pt}

{\bf{}Comparison Test 1}

\[I_1=\int_a^\infty f(x)\ dx\qquad I_2=\int_b^\infty g(x)\ dx\]

\[f(x)\leq g(x)\quad\mbox{for all}\quad x>M\]

If $I_1$ diverges, then $I_2$ diverges; if $I_2$ converges, then $I_1$ converges.

\begin{center}
\begin{tikzpicture}
\begin{axis}[standard,
xmin=-1, xmax=4,
ymin=0, ymax=2,
domain=-2:5,
xtick={\empty}, ytick={\empty}]
\addplot[blue] {-0.3*x+0.9+0.5};
\addplot[red] {-0.15*x+0.95};
\draw[dashed] (-1,0) -- (-1,2) node[pos=0,below]{$a$};
\draw[dashed] (3,0) -- (3,2) node[pos=0,below]{$M$};
\end{axis}
\end{tikzpicture}
\end{center}

\vspace{10pt}

{\bf{}EXAMPLE} $\displaystyle\int_0^\infty e^{-x^2}\ dx$

\[\mbox{for }x>1,\ x^2>x\Rightarrow e^{-x^2}<e^{-x}\]

\begin{center}
\begin{tikzpicture}
\begin{axis}[standard,
xmin=-1, xmax=4,
ymin=0, ymax=2,
domain=-2:5,
xtick={\empty}, ytick={\empty},
samples=50]
\addplot[blue] {e^(-x)};
\addplot[red] {e^(-x^2)};
\draw[dashed] (-1,0) -- (-1,2) node[pos=0,below]{$a$};
\draw[dashed] (3,0) -- (3,2) node[pos=0,below]{$M$};
\end{axis}
\end{tikzpicture}
\end{center}

\vspace{10pt}

So, $\int_0^\infty e^{-x^2}\ dx=\int_0^1 e^{-x^2}\ dx+\int_1^\infty e^{-x^2}\ dx=finite\ number\ +\ comparison\ test\ integral$

\[\int_1^\infty e^{-x}\ dx=e^{-1}\therefore\ integral\ is\ convergent\]

The following substitution is also useful,

\[\int_{-\infty}^cf(x)\ dx=(x=-t)=\int_{\infty}^{-c}f(-t)\ d(-t)=\int_{-c}^\infty f(-t)\ dt\]

\end{document}