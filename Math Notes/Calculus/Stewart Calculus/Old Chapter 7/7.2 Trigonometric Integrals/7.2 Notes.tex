\documentclass[a4]{article}

%other packages
\usepackage[a4paper, total={7.5in, 10.5in}]{geometry}
\usepackage{longtable}
\usepackage{wrapfig}
\setlength\parindent{0pt}
\usepackage{enumitem}
\usepackage[table]{xcolor}

%maths
\usepackage{mathtools}
\usepackage{amsmath}
\usepackage{amssymb}
\usepackage{amsfonts}
\usepackage{autobreak}

%tikzpicture
\usepackage{tikz}
\usepackage{scalerel}
\usepackage{pict2e}
\usepackage{tkz-euclide}
\usetikzlibrary{calc}
\usetikzlibrary{patterns,arrows.meta}
\usetikzlibrary{shadows}
\usetikzlibrary{external}

%pgfplots
\usepackage{pgfplots}
\pgfplotsset{compat=newest}
\usepgfplotslibrary{statistics}
\usepgfplotslibrary{fillbetween}

\begin{document}

\allowdisplaybreaks
\centerline{\bf{}Stewart, Outline \#7.2: Trigonometric Integrals}
{\renewcommand{\arraystretch}{2}
\begin{longtable}{|p{5cm}|p{7.78cm}|p{5cm}|}
\hline
    \rowcolor{lightgray}
    {\bf{}Outcome, you should be able to...} & {\bf{Show that you are able to do this.}} & {\bf{How will you not forget what you have learned?}}\\
\hline
State the strategy for evaluating $\int\sin^mx\cos^nx\ dx$ & 
\begin{enumerate}[label=(\alph*)]
\item If the power of cosine is odd, save one cosine factor and express rest i.t.o sine.
\item If the power of sine is odd, save one sine factor and express rest i.t.o cosine.
\item If the powers of both sine and cosine are even, use the half-angle identities.
\end{enumerate} & x\\
\hline
State the strategy for evaluating $\int\tan^mx\sec^nx\ dx$ & 
\begin{enumerate}[label=(\alph*)]
\item If the power of secant is odd, save one $sec^2x$ factor and express rest i.t.o tangent.
\item If the power of tangent is odd, save one $\sec x\tan x$ factor and express rest i.t.o secant.
\end{enumerate} & x\\
\hline
State the integral of tangent and derive the integral of secant. & $\int\tan x\ dx=\ln|\sec x|+C,\ \int\sec x\ dx=\ln|\sec x+\tan x|+C$ (p. 464) & x\\
\hline
State the product formulas. & $\sin x\cos y=\frac{1}{2}[\sin(x+y)+\sin(x-y)],\ \cos x\cos y=\frac{1}{2}[\cos(x+y)+\cos(x-y)],\ \sin x\sin y=\frac{1}{2}[\cos(x-y)-\cos(x+y)]$ & x\\
\hline
\end{longtable}
\end{document}