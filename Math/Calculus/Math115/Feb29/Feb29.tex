\documentclass{article}

%other packages
\usepackage[a4paper]{geometry}
\usepackage{longtable}
\usepackage{wrapfig}
\setlength\parindent{0pt}
\usepackage{enumitem}
\usepackage[table]{xcolor}
\usepackage{polynom}
\def\scaleint#1{\vcenter{\hbox{\scaleto[3ex]{\displaystyle\int}{#1}}}}
\usepackage{array}
\newcolumntype{C}{>{{}}c<{{}}} % for '+' and '-' symbols
\newcolumntype{R}{>{\displaystyle}r} % automatic display-style math mode 
\usepackage{tabularray}
\usepackage{dcolumn,tabularx,booktabs}
\usepackage{esvect}

%maths
\usepackage{mathtools}
\usepackage{amsmath}
\usepackage{amssymb}
\usepackage{amsfonts}
\usepackage{autobreak}

%tikzpicture
\usepackage{tikz}
\usepackage{scalerel}
\usepackage{pict2e}
\usepackage{tkz-euclide}
\usepackage{tikz-3dplot}
\usetikzlibrary{calc}
\usetikzlibrary{patterns,arrows.meta}
\usetikzlibrary{shadows}
\usetikzlibrary{external}
\usetikzlibrary{decorations.pathreplacing,angles,quotes}
\usetikzlibrary{perspective,spath3}

%pgfplots
\usepackage{pgfplots}
\pgfplotsset{compat=1.18}
\usepgfplotslibrary{statistics}
\usepgfplotslibrary{fillbetween}

\pgfplotsset{
    standard/.style={
    axis line style = thick,
    trig format=rad,
    enlargelimits,
    axis x line=middle,
    axis y line=middle,
    enlarge x limits=0.15,
    enlarge y limits=0.15,
    every axis x label/.style={at={(current axis.right of origin)},anchor=north west},
    every axis y label/.style={at={(current axis.above origin)},anchor=south east}
    }
}

\begin{document}

Math 115 - Partial Fractions - 29 Feb 2024
\hrule

\vspace{10pt}

Integration by partial fractions is based on one basic concept. If you have a quotient of two polynomials where the numerator has a degree that is less than the denominator, then you can break it apart into a sum of fractions - one for each factor of the denominator. Basically, we do the following formula backwards.

\[\frac{a}{b}+\frac{c}{d}=\frac{ad+cb}{bd}\]

If the numerator's degree is equal to or greater than the denominator's, then we first do polynomial long division:

\[\setlength\arraycolsep{0pt} 
\setlength\extrarowheight{2pt}
\begin{array}[t]{ RCCRCRCRCR }
	& & x^2 & - & x & - & 6 & & &\\
\cline{2-9}
	x-2 & \kern-0.4pt\raisebox{1.6pt}{\big)} & x^3 & - & 3x^2 & - & 4x & + & 12 \\
	& & x^3 & - & 2x^2 & & & & & \\
\cline{3-7}
	& & & - & x^2 & - & 4x & & & \\
	& & & - & x^2 & + & 2x & & & \\
\cline{4-9}
	& & & & & - & 6x & + & 12 & \\
	& & & & & - & 6x & + & 12 & \\
\cline{6-10}
	& & & & & & & & 0 &
\end{array}\]

If this is hard, I cannot more highly recommend Stewart's Review of Algebra (free download).

\vspace{10pt}

There are four cases that we will look at. And I need to mention that they are not exclusive of eachother - one integrand may be a combination of a few. The first one is where the denominator is comprised of linear coefficients (of degree 1!), and we split into a sum of fractions where the numerators are constants. The second is for linear factors of degrees higher than one; essentially, we add one fraction for each. Each of our partial fractions will have that linear raised to an exponent - from 1, to the power of the of the original linear expression. For example;

\[\frac{7x+3}{(5x+6)^5}=\frac{C_1}{(5x+6)^1}+\cdots+\frac{C_5}{(5x+6)^5}\]

The third and fourth cases are analygous, except their denominators are either repeated or non-repeated irreducible quadratics, and the numerator becomes a linear instead of a constant. For case 3, when we get our partial fractions, we complete the square in the denominator and use u-substitution and the derivative of arctan to get our answer. For case 4, we can substitute $u=$irreducible quadratic, and make the numerator $du$

\vspace{10pt}

{\bf{}CASE: 1} linear factors of degree 1

\[\frac{c_1x^n+\cdots+c_nx^0}{(a_1x+b_1)\times\cdots\times(a_{m\geq n}x+b_{m\geq n})}=\frac{A_1}{a_1x+b_1}+\cdots+\frac{A_{m\geq n}}{(a_{m\geq n}x+b_{m\geq n})}\]

*solve with $\int\frac{a}{bx+c}\ dx=a\ln|bx+c|+C$

\vspace{10pt}

{\bf{}CASE: 2} linear factors of degree greater than 1

\[\frac{c_1x^n+\cdots+c_nx^0}{(a_1x+b_1)^{m\geq n}}=\frac{A_1}{a_1x+b_1}+\cdots+\frac{A_m}{(a_1x+b_1)^m}\]

*solve with $\int\frac{a}{bx+c}\ dx=a\ln|bx+c|+C$ and u-substitution to make integrals of the same form

{\bf{}CASE: 3} irreducible quadratic factors of degree 1

\[\frac{a_1x^0+\cdots+a_{2n}x^{2n-1}}{(\mbox{quadratic}_1)\times\cdots\times(\mbox{quadratic}_n)}=\frac{A_1x+B_1}{\mbox{quadratic}_1}+\cdots+\frac{A_nx+B_n}{\mbox{quadratic}_n}\]

*solve using u-substitution, and/or completing the square followed by making use of the derivative of arctan

\vspace{10pt}

{\bf{}CASE: 4} irreducible quadratic factors of degree greater than 1

\[\frac{a_1x^0+\cdots a_{2n-1}x^{2n-1}}{(\mbox{quadratic})^n}=\frac{A_1x+B_1}{\mbox{quadratic}}+\cdots+\frac{A_nx+B_n}{(\mbox{quadratic})^n}\]

\vspace{10pt}

Selected Questions:

\begin{enumerate}
\item[39.] $\displaystyle\int\frac{1}{(x\sqrt{x+1})}\ dx$
\begin{align*}
\int\frac{1}{(x\sqrt{x+1})}\ dx&=2\int\frac{1}{x}\cdot\frac{a}{2\sqrt{x+1}}\ dx\\
&=(u=\sqrt{x+1})\\
&=2\int\frac{1}{u^2-1}\ du\\
&=\mbox{ trivial}
\end{align*}
\item[40.] $\displaystyle\int\frac{dx}{2\sqrt{x+3}+x}$
\begin{align*}
\int\frac{dx}{2\sqrt{x+3}+x}&=(u=\sqrt{x+3})\\
&=\int\frac{2u\ du}{2u+u^2-3}\\
&=\mbox{ trivial}
\end{align*}
\item[41.] $\displaystyle\int\frac{\sqrt{x}}{x-4}\ dx$
\begin{align*}
\int\frac{\sqrt{x}}{x-4}\ dx&=(u=\sqrt{x})\\
&=\int\frac{2u^2\ du}{u^2-4}\\
&=\mbox{ trivial}
\end{align*}
\item[42.] $\displaystyle\int\frac{1}{1+\sqrt[3]{x}}\ dx$
\begin{align*}
\int\frac{1}{1+\sqrt[3]{x}}\ dx&=(u=\sqrt[3]{x})\\
&=\int\frac{3u^2\ du}{1+u}\\
&=\mbox{ trivial}
\end{align*}
\item[43.] $\displaystyle\int\frac{x^3}{\sqrt[3]{x^2+1}}\ dx$
\begin{align*}
\int\frac{x^3}{\sqrt[3]{x^2+1}}\ dx&=\int\frac{x^2}{\sqrt[3]{x^2+1}}\ xdx\\
&=(u=\sqrt[3]{x^2+1};\ du=2x\ dx)\\
&=\frac{1}{2}\int\frac{u^3-1}{u}\ du\\
&=\mbox{ trivial}
\end{align*}
\item[46.] $\displaystyle\int\frac{\sqrt{1+\sqrt{x}}}{x}\ dx$
\begin{align*}
\int\frac{\sqrt{1+\sqrt{x}}}{x}\ dx&=(u=\sqrt{x};\ u^2=x;\ dx=2u\ du)\\
&=2\int\frac{\sqrt{1+u}}{u^2}\ du\\
&=\mbox{ trivial after onemore rationalizing substitution}
\end{align*}
\end{enumerate}






\end{document}