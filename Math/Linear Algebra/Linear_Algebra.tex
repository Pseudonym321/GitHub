\documentclass{article}

%other packages
\usepackage[a4paper]{geometry}
\usepackage{longtable}
\usepackage{wrapfig}
\setlength\parindent{0pt}
\usepackage{enumitem}
\usepackage[table,dvipsnames]{xcolor}
\usepackage{polynom}
\def\scaleint#1{\vcenter{\hbox{\scaleto[3ex]{\displaystyle\int}{#1}}}}
\usepackage{array}
\newcolumntype{C}{>{{}}c<{{}}} % for '+' and '-' symbols
\newcolumntype{R}{>{\displaystyle}r} % automatic display-style math mode 
\usepackage{tabularray}
\usepackage{dcolumn,tabularx,booktabs}
\usepackage[most]{tcolorbox}
%\graphicspath{ {C:/Users/twill/OneDrive/Desktop/Eliason/Diagrams} }

%maths
\usepackage{mathtools}
\usepackage{amsmath}
\usepackage{amssymb}
\usepackage{amsfonts}
\usepackage{autobreak}

%tikzpicture
\usepackage{tikz}
\usepackage{scalerel}
\usepackage{pict2e}
\usepackage{tkz-euclide}
\usepackage{tikz-3dplot}
\usetikzlibrary{calc}
\usetikzlibrary{patterns,arrows.meta}
\usetikzlibrary{shadows}
\usetikzlibrary{external}
\usetikzlibrary{decorations.pathreplacing,angles,quotes}

%pgfplots
\usepackage{pgfplots}
\pgfplotsset{compat=1.18}
\usepgfplotslibrary{statistics}
\usepgfplotslibrary{fillbetween}

\pgfplotsset{
    standard/.style={
    axis line style = thick,
    trig format=deg,
    enlargelimits,
    axis x line=middle,
    axis y line=middle,
    enlarge x limits=0.15,
    enlarge y limits=0.15,
    every axis x label/.style={at={(current axis.right of origin)},anchor=north west},
    every axis y label/.style={at={(current axis.above origin)},anchor=south east}
    }
}

\begin{document}

1.1 Solving Linear Systems
\hrule

\vspace{10pt}

A system of equations is a group of more than one equation which are simultaneously true; they can be represented as an array as follows:\\

\begin{center}
$\boxed{\begin{aligned}
a_1,{}_1x_1+a_1,{}_2x_2+\cdots+a_1,{}_nx_n&=d_1\\
a_2,{}_1x_1+a_2,{}_2x_2+\cdots+a_2,{}_nx_n&=d_2\\
&\hspace{5pt} \vdots\\
a_m,{}_1x_1+a_m,{}_2x_2+\cdots+a_m,{}_nx_n&=d_m
\end{aligned}}$
\end{center}

An $n$-tuple $(s_1,s_2,\ldots,s_n)\in\mathbb{R}^n$ satisfies a system if it solves each equation.\\

$\boxed{\begin{aligned}
\multicolumn{1}{p{5.5in}}{{\bf{}Theorem (Gauss's Method)} If a linear system is changed to another by one of these operations}\\
\multicolumn{1}{l}{\phantom{M}(1) an equation is swapped with another}\\
\multicolumn{1}{l}{\phantom{M}(2) an equation has both sides multiplied by a nonzero constant}\\
\multicolumn{1}{l}{\phantom{M}(3) an equation is replaces by the sum of itselt and a multiple of another}\\
\multicolumn{1}{p{5.5in}}{then the two systems have the same of solutions.}
\end{aligned}}$\\

{\bf{}Proof} For the swap operation, since it is just a conjunction of true statements, and conjunction is commutative, it must also be commutative.\\

$\boxed{\begin{aligned}
\multicolumn{2}{l}{The row operations are indicated as follows, where $\rho$ (rho) means row.}\\
swapping&\quad\rho_n\leftrightarrow\rho_m\\
rescaling&\quad\alpha\rho_n\\
combination&\quad\alpha\rho_n+\rho_m\mbox{ where the first is added to the second}\\
\end{aligned}}$\\

A system is in echelon form if each leading variable is to the right of the leading variable in the row above it, except for the leading variable in the first row, and any rows with all-zero coefficients are at the bottom.\\

In summary, Gauss’s Method uses the row operations to set a system up for
back substitution. If any step shows a contradictory equation then we can stop
with the conclusion that the system has no solutions. If we reach echelon form
without a contradictory equation, and each variable is a leading variable in its
row, then the system has a unique solution and we find it by back substitution.
Finally, if we reach echelon form without a contradictory equation, and there is
not a unique solution—that is, at least one variable is not a leading variable—
then the system has many solutions.

\newpage

{\bf{}Exercises}

\begin{enumerate}
\item[1.17] Use Gauss's Method to find the unique solution for each system.
\begin{enumerate}

\hrule

\item[(a)]

$\begin{aligned}
2x+3y&=13\\
x-y&=-1
\end{aligned}
\overset{-2\rho_2+\rho_1}{\longrightarrow}
\begin{aligned}
0x+5y&=15\\
x-y&=-1
\end{aligned}
\overset{\frac{1}{5}\rho_1}{\longrightarrow}
\begin{aligned}
0x+y&=3\\
x-y&=-1
\end{aligned}
\overset{\rho_1+\rho_2}{\longrightarrow}
\begin{aligned}
0x+y&=3\\
x+0y&=2
\end{aligned}$

\hrule

\item[(b)]

$\begin{aligned}
x&&&\ -&z&=&0\\
3x&\ +&y&&&=&1\\
-x&\ +&y&\ +&z&=&4
\end{aligned}
\overset{\rho_1+\rho_3}{\longrightarrow}
\begin{aligned}
x&&&\ -&z&=&0\\
3x&\ +&y&&&=&1\\
&&y&&&=&4
\end{aligned}
\overset{-\rho_3+\rho_2}{\longrightarrow}
\begin{aligned}
x&&&\ -&z&=&0\\
3x&&&&&=&-3\\
&&y&&&=&4
\end{aligned}$

\hrule

$\overset{\frac{1}{3}\rho_2}{\longrightarrow}
\begin{aligned}
x&&&\ -&z&=&0\\
x&&&&&=&-1\\
&&y&&&=&4
\end{aligned}
\overset{-\rho_2+\rho_1}{\longrightarrow}
\begin{aligned}
&&&\ -&z&=&1\\
x&&&&&=&-1\\
&&y&&&=&4
\end{aligned}
\overset{-\rho_1}{\longrightarrow}
\begin{aligned}
&&&&z&=&-1\\
x&&&&&=&-1\\
&&y&&&=&4
\end{aligned}$

\hrule

\end{enumerate}

\item[1.18] Each system is in echelon form. For each, say whether the system has a unique solution, no solution, or infinitely many solutions.

\begin{enumerate}
\item[(a)]
$\begin{aligned}
-3x&+&2y&=&0\\
&&-2y&=&0
\end{aligned}
\overset{\rho_2+\rho_1}{\longrightarrow}
\begin{aligned}
-3x&&&=&0\\
&&-2y&=&0
\end{aligned}
\underset{-2\rho_2}{\overset{-3\rho_1}{\longrightarrow}}
\begin{aligned}
x&&&=&0\\
&&y&=&0
\end{aligned}$

The system has a unique solution since each term can be assigned a fixed, non-parametric, value.

\item[(b)]

$\begin{aligned}
x&\ +&y&&&=&4\\
&&y&\ -&z&=&0
\end{aligned}
\overset{-\rho_2+\rho_1}{\longrightarrow}
\begin{aligned}
x&&&+&z&=&4\\
&&y&\ -&z&=&0
\end{aligned}
\overset{\rho_1+\rho_2}{\longrightarrow}
\begin{aligned}
x&&&+&z&=&4\\
x&+&y&&&=&4
\end{aligned}$

$\overset{-\rho_1+\rho_2}{\longrightarrow}
\begin{aligned}
x&&&+&z&=&4\\
&+&y&-&z&=&0
\end{aligned}$

There is no way to isolate the parameter $z$, therefore there are infinitely many solutions. That is, there are variable(s) that do not lead a row (i.e. $z$).

\item[(c)]

$\begin{aligned}
x&\ +&y&&&=&4\\
&&y&\ -&z&=&0\\
&&&&0&=&0
\end{aligned}
\overset{-\rho_1+\rho_2}{\longrightarrow}
\begin{aligned}
x&\ +&y&&&=&4\\
-x&&&\ -&z&=&-4\\
&&&&0&=&0
\end{aligned}
\overset{\rho_2+\rho_1}{\longrightarrow}
\begin{aligned}
&&y&\ +&z&=&0\\
-x&&&\ -&z&=&-4\\
&&&&0&=&0
\end{aligned}$

$\overset{\rho_1+\rho_2}{\longrightarrow}
\begin{aligned}
&&y&\ +&z&=&0\\
-x&\ +&y&&&=&-4\\
&&&&0&=&0
\end{aligned}
\overset{-\rho_2+\rho_1}{\longrightarrow}
\begin{aligned}
x&&&\ +&z&=&0\\
-x&\ +&y&&&=&-4\\
&&&&0&=&0
\end{aligned}$

The variable $y$ does not lead a row, therefore there are infinitely many solutions.

\item[(d)]

$\begin{aligned}
x&+&y&=&4\\
&&0&=4
\end{aligned}$

The system is inconsistent, given that $0\neq4$. Therefore, there are no solutions.

\item[(e)]

$\begin{aligned}
3x&\ +&6y&\ +&z&=&-0.5\\
&&&&-z&=&2.5
\end{aligned}$

There is no way to isolate $x$ and $y$ separately, therefore there are infinitely many solutions.

\item[(f)]

$\begin{aligned}
x&\ -&3y&=&2\\
&&0&=&0
\end{aligned}$

There is no way to separately isolate $x$ and $y$, and the system is not inconsistent, therefore there are infinitely many solutions.

\item[(g)]

$\begin{aligned}
2x&\ +&2y&=&4\\
&&y&=&1\\
&&0&=&4
\end{aligned}$

The system is inconsistent because $0\neq4$, therefore there are zero solutions.

\item[(h)]

$2x+y=0$ has $\infty$ solutions because it is not inconsistent, and neither variable can be isolated.

\item[(i)]

$\begin{aligned}
x&\ -&y&=&-1\\
&&0&=&0\\
&&0&=&4
\end{aligned}$

The system is inconsistent because $0\neq4$, therefore there are zero solutions.

\item[(j)]

$\begin{aligned}
x&\ +&y&\ -&3z&=&-1\\
&&y&\ -&z&=&2\\
&&&&z&=&0\\
&&&&0&=&0
\end{aligned}
\underset{3\rho_3+\rho_1}{\overset{\rho_3+\rho_2}{\longrightarrow}}
\begin{aligned}
x&\ +&y&&&=&-1\\
&&y&&&=&2\\
&&&&z&=&0\\
&&&&0&=&0
\end{aligned}
\overset{-\rho_2+\rho_1}{\longrightarrow}
\begin{aligned}
x&&&&&=&-3\\
&&y&&&=&2\\
&&&&z&=&0\\
&&&&0&=&0
\end{aligned}$

The system is not inconsistent, and there is a unique value for each variable. As such, it has one solution.
\end{enumerate}

\item[1.20]

\begin{enumerate}

\item[(a)]

$\begin{aligned}
x&\ +&y&\ +&z&=&5\\
x&\ -&y&&&=&0\\
&&y&\ +&2z&=&7
\end{aligned}
\overset{\rho_2+\rho_1}{\longrightarrow}
\begin{aligned}
2x&&&\ +&z&=&5\\
x&\ -&y&&&=&0\\
&&y&\ +&2z&=&7
\end{aligned}
\overset{\rho_3+\rho_2}{\longrightarrow}
\begin{aligned}
2x&&&\ +&z&=&5\\
x&&&&&2z=&7\\
&&y&\ +&2z&=&7
\end{aligned}$

\hrule

$\overset{-2\rho_2+\rho_1}{\longrightarrow}
\begin{aligned}
&&&&-3z&=&-9\\
x&&&&&2z=&7\\
&&y&\ +&2z&=&7
\end{aligned}
\overset{-\frac{1}{3}\rho_1}{\longrightarrow}
\begin{aligned}
&&&&z&=&3\\
x&&&&&2z=&7\\
&&y&\ +&2z&=&7
\end{aligned}$

\hrule

$\overset{-2\rho_1+\rho_2}{\longrightarrow}
\begin{aligned}
&&&&z&=&3\\
x&&&&&=&1\\
&&y&\ +&2z&=&7
\end{aligned}
\overset{-2\rho_1+\rho_3}{\longrightarrow}
\begin{aligned}
&&&&z&=&3\\
x&&&&&=&1\\
&&y&&&=&1
\end{aligned}$

There are no inconsistencies, and each variable has a unique solution. Therefore, the system has a unique solution.

\item[(b)]

$\begin{aligned}
3x&&&\ +&z&=&7\\
x&\ -&y&\ +&3z&=&4\\
x&\ +&2y&\ -&5z&=&-1
\end{aligned}
\overset{-3\rho_1+\rho_2}{\longrightarrow}
\begin{aligned}
3x&&&\ +&z&=&7\\
-8x&\ -&y&&&=&-17\\
x&\ +&2y&\ -&5z&=&-1
\end{aligned}$

\hrule

$\overset{2\rho_2+\rho_3}{\longrightarrow}
\begin{aligned}
3x&&&\ +&z&=&7\\
-8x&\ -&y&&&=&-17\\
-15x&&&\ -&5z&=&-35
\end{aligned}
\overset{5\rho_1+\rho_3}{\longrightarrow}
\begin{aligned}
3x&&&\ +&z&=&7\\
-8x&\ -&y&&&=&-17\\
&&&&0&=&0
\end{aligned}$

\hrule

There are more variables than equations, therefore the system has $\infty$ solutions.

\item[(c)]

\hrule

$\begin{aligned}
x&\ +&3y&\ +&z&=&0\\
-x&\ -&y&&&=&2\\
-x&\ +&y&\ +&2z&=&8
\end{aligned}
\overset{3\rho_2+\rho_1}{\longrightarrow}
\begin{aligned}
-2x&&&\ +&z&=&6\\
-x&\ -&y&&&=&2\\
-x&\ +&y&\ +&2z&=&8
\end{aligned}
\overset{\rho_2+\rho_3}{\longrightarrow}
\begin{aligned}
-2x&&&\ +&z&=&6\\
-x&\ -&y&&&=&2\\
-2x&&&\ +&2z&=&10
\end{aligned}$

\hrule

$\overset{-\rho_1+\rho_3}{\longrightarrow}
\begin{aligned}
-2x&&&\ +&z&=&6\\
-x&\ -&y&&&=&2\\
&&&&z&=&4
\end{aligned}
\overset{-\rho_3+\rho_1}{\longrightarrow}
\begin{aligned}
-2x&&&&&=&2\\
-x&\ -&y&&&=&2\\
&&&&z&=&4
\end{aligned}
\overset{-2\rho_1}{\longrightarrow}
\begin{aligned}
x&&&&&=&-1\\
-x&\ -&y&&&=&2\\
&&&&z&=&4
\end{aligned}$

\hrule

$\overset{\rho_1+\rho_2}{\longrightarrow}
\begin{aligned}
x&&&&&=&-1\\
&\ -&y&&&=&1\\
&&&&z&=&4
\end{aligned}
\overset{-\rho_2}{\longrightarrow}
\begin{aligned}
x&&&&&=&-1\\
&&y&&&=&-1\\
&&&&z&=&4
\end{aligned}$

There are no inconsistencies, and each variable has been assigned a specific value; as such, the system has a unique solution.

\end{enumerate}

\item[1.21] A commonly taught method for solving linear systems is:

\begin{itemize}

\item[1.] Solve one equation for one variable.

\item[2.] Substitute the resulting expression back into the other equations.

\item[3.] Repeat until you find a value for a variable.

\item[4.] Back substitute until you get the answer.

\end{itemize}

This can lead to apparent inconsistencies when a variable can seem to have multiple values. Before believing there is a solution, you must back substitute all solved values into the original (or equivalent) system. Otherwise, you just solved a variety of 2-systems.

\item[1.22] For which values of $k$ are there no solutions, many solutions, or a unique solution to this system?

\[\begin{aligned}
x&\ -&y&=&1\\
3x&\ -&3y&=&k
\end{aligned}\]

\hrule

$\begin{aligned}
x&\ -&y&=&1\\
3x&\ -&3y&=&k
\end{aligned}
\overset{3\rho_1+\rho_2}{\longrightarrow}
\begin{aligned}
x&\ -&y&=&1\\
&&0&=&k-3
\end{aligned}$

\hrule

There are many solutions if $k=3$, and no solutions for any other value. There are no values of $k$ for which there is a unique solution.

\item[1.23]Solve the following non-linear system using Gauss's method.

\[\begin{aligned}
2\sin\alpha&\ -&\cos\beta&\ +&3\tan\gamma&=&3\\
4sin\alpha&\ +&2\cos\beta&\ -&2\tan\gamma&=&10\\
6\sin\alpha&\ -&3\cos\beta&\ +&\tan\gamma&=&9
\end{aligned}\]

\hrule

Let $x=\sin\alpha$, $y=\cos\beta$ and $z=\tan\gamma$

\hrule

$\begin{aligned}
2x&\ -&y&\ +&3z&=&3\\
4x&\ +&2y&\ -&2y&=&10\\
6x&\ -&3y&\ +&z&=&9
\end{aligned}
\overset{-3\rho_1+\rho_3}{\longrightarrow}
\begin{aligned}
2x&\ -&y&\ +&3z&=&3\\
4x&\ +&2y&\ -&2z&=&10\\
&&&\ -&8z&=&0
\end{aligned}
\overset{-8\rho_3}{\longrightarrow}
\begin{aligned}
2x&\ -&y&\ +&3z&=&3\\
4x&\ +&2y&\ -&2z&=&10\\
&&&&z&=&0
\end{aligned}$

\hrule

$\underset{-3\rho_3+\rho_1}{\overset{2\rho_3+\rho_2}{\longrightarrow}}
\begin{aligned}
2x&\ -&y&&&=&3\\
4x&\ +&2y&&&=&10\\
&&&&z&=&0
\end{aligned}
\overset{2\rho_1+\rho_2}{\longrightarrow}
\begin{aligned}
2x&\ -&y&&&=&3\\
8x&&&&&=&16\\
&&&&z&=&0
\end{aligned}
\overset{\frac{1}{8}\rho_2}{\longrightarrow}
\begin{aligned}
2x&\ -&y&&&=&3\\
x&&&&&=&2\\
&&&&z&=&0
\end{aligned}$

\hrule

$\overset{-2\rho_2+\rho_1}{\longrightarrow}
\begin{aligned}
&\ -&y&&&=&-1\\
x&&&&&=&2\\
&&&&z&=&0
\end{aligned}
\overset{-\rho_1}{\longrightarrow}
\begin{aligned}
&&y&&&=&1\\
x&&&&&=&2\\
&&&&z&=&0
\end{aligned}$

\end{enumerate}




\end{document}