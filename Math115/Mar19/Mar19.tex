\documentclass{article}

%other packages
\usepackage[a4paper]{geometry}
\usepackage{longtable}
\usepackage{wrapfig}
\setlength\parindent{0pt}
\usepackage{enumitem}
\usepackage[table,dvipsnames]{xcolor}
\usepackage{polynom}
\def\scaleint#1{\vcenter{\hbox{\scaleto[3ex]{\displaystyle\int}{#1}}}}
\usepackage{array}
\newcolumntype{C}{>{{}}c<{{}}} % for '+' and '-' symbols
\newcolumntype{R}{>{\displaystyle}r} % automatic display-style math mode 
\usepackage{tabularray}
\usepackage{dcolumn,tabularx,booktabs}
\usepackage[most]{tcolorbox}
%\graphicspath{ {C:/Users/twill/OneDrive/Desktop/Eliason/Diagrams} }

%maths
\usepackage{mathtools}
\usepackage{amsmath}
\usepackage{amssymb}
\usepackage{amsfonts}
\usepackage{autobreak}

%tikzpicture
\usepackage{tikz}
\usepackage{scalerel}
\usepackage{pict2e}
\usepackage{tkz-euclide}
\usepackage{tikz-3dplot}
\usetikzlibrary{calc}
\usetikzlibrary{patterns,arrows.meta}
\usetikzlibrary{shadows}
\usetikzlibrary{external}
\usetikzlibrary{decorations.pathreplacing,angles,quotes}

%pgfplots
\usepackage{pgfplots}
\pgfplotsset{compat=1.18}
\usepgfplotslibrary{statistics}
\usepgfplotslibrary{fillbetween}

\pgfplotsset{
    standard/.style={
    axis line style = thick,
    trig format=deg,
    enlargelimits,
    axis x line=middle,
    axis y line=middle,
    enlarge x limits=0.15,
    enlarge y limits=0.15,
    every axis x label/.style={at={(current axis.right of origin)},anchor=north west},
    every axis y label/.style={at={(current axis.above origin)},anchor=south east}
    }
}

\begin{document}

Math 115, 19 March 2024
\hrule

\vspace{10pt}

\begin{equation}\boxed{\int_0^a\frac{1}{x^P}\ dx\begin{aligned}\mbox{convg.}\ p<1\\\mbox{divg.}\ p\geq1\end{aligned}}\end{equation}

\begin{equation}\boxed{\int_1^\infty\frac{1}{x^P}\ dx\begin{aligned}\mbox{convg.}\ p>1\\\mbox{divg.}\ p\leq1\end{aligned}}\end{equation}

For $f(x)\leq g(x)$, where $0<x<m$, if $I_f=\int_0^1f(x)\ dx\quad I_g=\int_0^1g(x)$, then; (a) if $I_g$ convg., then $I_f$ is too, and (b) if $I_g$ digv., then $I_f$ is too.

\vspace{10pt}

{\bf{}EXAMPLE} $\int_0^1\frac{4+\sin x}{x\sqrt{x}}\ dx\leq\int_0^1\frac{2}{x^{3/2}}\ dx$

\vspace{10pt}

{\bf{}EXAMPLE} $\int_0^1\frac{\sin x}{x\sqrt x}\ dx$

\[\sin x\sim x\implies\frac{\sin x}{x\sqrt{x}}\sim\frac{1}{\sqrt{x}}\]

\[\lim_{x\to0^+}\left(\frac{\sin x}{x\sqrt{x}}\div\frac{1}{\sqrt{x}}\right)=1\]

{\bf{}EXAMPLE} $\int_a^\infty\frac{x\arctan x}{(1+x^2)^2}\ dx$

\vspace{10pt}

$\arctan x\leq\pi/2$ as $x\to\infty$.

\begin{align*}
\int_a^\infty\frac{x\arctan x}{(1+x^2)^2}\ dx&\leq\frac{\pi}{2}\int_a^\infty\frac{x}{(1+x^2)^2}\ dx\\
&=\frac{\pi}{2}\int_0^\infty\frac{0.5\ d(1+x^2)}{(1+x^2)^2}\\
&\Rightarrow\mbox{ convg.}
\end{align*}

Knowing that it is convergent, we can evaluate it with integration by parts; $u=\arctan x$

\vspace{10pt}

{\bf{}EXAMPLE} $\int_{\pi/2}^\pi\frac{1}{\sin x}\ dx$

\begin{align*}
\int_{\pi/2}^\pi\frac{1}{\sin x}\ dx&=\left(\begin{aligned}x=\pi-t&\quad x=\pi\implies t=0\\dx=-dt&\quad x=\frac{\pi}{2}\implies t=\frac{\pi}{2}\end{aligned}\right)\\
&=-\int_\frac{\pi}{2}^\pi\frac{1}{\sin t}\ dt=\int_0^\frac{\pi}{2}\frac{1}{\sin t}\ dt\\
\frac{1}{\sin t}&\sim\frac{1}{t}\mbox{ when }t\to0^+\\
&\therefore\mbox{ divergent by comparison.}
\end{align*}







\end{document}
