\documentclass{article}

%other packages
\usepackage[a4paper]{geometry}
\usepackage{longtable}
\usepackage{wrapfig}
\setlength\parindent{0pt}
\usepackage{enumitem}
\usepackage[table,dvipsnames]{xcolor}
\usepackage{polynom}
\def\scaleint#1{\vcenter{\hbox{\scaleto[3ex]{\displaystyle\int}{#1}}}}
\usepackage{array}
\newcolumntype{C}{>{{}}c<{{}}} % for '+' and '-' symbols
\newcolumntype{R}{>{\displaystyle}r} % automatic display-style math mode 
\usepackage{tabularray}
\usepackage{dcolumn,tabularx,booktabs}
\usepackage[most]{tcolorbox}

%maths
\usepackage{mathtools}
\usepackage{amsmath}
\usepackage{amssymb}
\usepackage{amsfonts}
\usepackage{autobreak}

%tikzpicture
\usepackage{tikz}
\usepackage{scalerel}
\usepackage{pict2e}
\usepackage{tkz-euclide}
\usepackage{tikz-3dplot}
\usetikzlibrary{calc}
\usetikzlibrary{patterns,arrows.meta}
\usetikzlibrary{shadows}
\usetikzlibrary{external}
\usetikzlibrary{decorations.pathreplacing,angles,quotes}

%pgfplots
\usepackage{pgfplots}
\pgfplotsset{compat=1.18}
\usepgfplotslibrary{statistics}
\usepgfplotslibrary{fillbetween}

\pgfplotsset{
    standard/.style={
    axis line style = thick,
    trig format=deg,
    enlargelimits,
    axis x line=middle,
    axis y line=middle,
    enlarge x limits=0.15,
    enlarge y limits=0.15,
    every axis x label/.style={at={(current axis.right of origin)},anchor=north west},
    every axis y label/.style={at={(current axis.above origin)},anchor=south east}
    }
}

\begin{document}

Math 115 - Week 4, Class 11 - 26 Jan 2024
\hrule

\vspace{10pt}

A question was asked at the beginning of; the answer was that we can ignore the exact value of, for example, $\arctan(-2)$ in situations where we are asked to take the inverse of the operation. That is, $\tan(\arctan(-2))=-2$ and it doesn't matter what $\arctan(-2)$ equals - I will omit the specific question though as I don't want this to be confusing.

\vspace{10pt}

{\bf{}EXAMPLE} Evaluate $\displaystyle\int\frac{1+x}{1+x^2}\ dx$

\vspace{10pt}

We must immediately recognize this as a sum of the derivative of the tangent function and another function which is integrable by substitution - and a reciprocal integration, though this is something we'd probable determine after the obvious substitution. By obvious, I mean that we could do it with a table integral.

\begin{align*}
\int\frac{1+x}{1+x^2}\ dx&=\arctan x+\int\frac{x}{1+x^2}\ dx\\
&=\left(\begin{array}{c}x\ dx=\frac{1}{2}\ d(x^2+1)\\d(x^2+1)=2x\ dx\end{array}\right.\\
\mbox{Let }u&=x^2+1\Rightarrow d\ dx=\frac{1}{2}\ du\\
&=\arctan(x)+\frac{1}{2}\ln|u|+C\\
&=\arctan(x)+\frac{1}{2}\ln(x^2+1)+C
\end{align*}

\vspace{10pt}

{\bf{}EXAMPLE} Evaluate $\displaystyle\int\cos x\cdot3^{\sin x}\ dx$

\vspace{10pt}

Always look for derivative-antiderivative pairs when integrating functions; in this case, this substitution is also "obvious." The reason these kinds of derivative-antiderivative substitutions work is because of an advanced concept called "invariance of the form of the first differential." We are not expected to know how this works - remotely; we were just made aware of its existence.

\vspace{10pt}

And by letting $u=\sin x$, we get $\int\cos x\cdot3^{\sin x}\ dx=3^{\sin x}/\ln3+C$.

\vspace{10pt}

A question was asked regarding one of the homework problems, $\tan(\arctan\frac{1}{2}+\arccos\frac{1}{3}$. Essentially, we rewrite cosine in terms of tangent, then use the tangent sum formula.

\vspace{10pt}

{\bf{}IMPORTANT} Dr. Solomonovich indicated that there will be a related-rates question on the test - related to the related rates of the sidelengths of a right triangle in the context of finding out the horizontal velocity of an ascending air-plane. Essentially, we implicitly differentiate - with respect to time - a function which relates our quantities of interest, then plug in whatever values that are provided by the question.

\vspace{10pt}

We need to know the definitions of hyperbolic sine cosine and tangent - hyperbolic tangent has three equivalent definitions which can be transitioned between using the algebraic manipulations mentioned last class. The one that is easiest to graph is $\tanh x=(e^{2x}-1)/(e^{2x}+1)$

\vspace{10pt}

We can prove that the tangent of hyperbolic sine at the origin is one;

\begin{align*}
(\sinh x)^\prime&=\cosh x\\
\mbox{So, }(\sinh x)^\prime\big|_{x=0}&=1\\
\mbox{and }\cosh x\big|_{x=0}&=1\\
\end{align*}

\vspace{10pt}

We can prove that hyperbolic cosine is horizontal when $x=0$ using the same method. We are expected to know the graphs and long-term behavior of the hyperbolic functions. Notable is that hyperbolic sine and cosine are "curvilinear asymptotes" of eachother - in the positive direction. That is, they approach each other asymptotically. We can prove this by taking the limit of their quotient as $x$ becomes arbitrarily large and getting $1$ as our answer.

\begin{center}
\begin{tikzpicture}
\begin{axis}[standard]
\addplot[samples=100] {tanh(x)} node[pos=0.9,below]{$y=\tanh x$};

\end{axis}
\end{tikzpicture}
\end{center}

\vspace{10pt}

We need to be able to define the inverse hyperbolic parent functions, as they are important in certain types of integration. For example; sine hyperbolic:

\begin{align*}
y&=\mbox{arcsinh } x\\
x&=\sinh y\\
2x&=e^y+e^{-y}\\
e^y\cdot2x&=e^{2y}+1\\
e^{2y}-2xe^y-1&=0
\end{align*}

\vspace{10pt}

Therefore, by the quadratic formula $e^y=x+\sqrt{x^2+1}\Rightarrow y=\ln(x+\sqrt{x^2+1})$

\vspace{10pt}

{\bf{}HOMEWORK} Derive the formulas for the inverse hyperbolic tangent and cosine functions.

\vspace{10pt}

We will not be given trick questions (eg. find $\mbox{arctanh }(3)$) on the test. All questions will actually have an answer. This question may be rephrased as "find $\mbox{arctanh }\frac{1}{3}$."












\end{document}