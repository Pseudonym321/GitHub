\documentclass{article}

%other packages
\usepackage[a4paper]{geometry}
\usepackage{longtable}
\usepackage{wrapfig}
\setlength\parindent{0pt}
\usepackage{enumitem}
\usepackage[table]{xcolor}
\usepackage{polynom}
\def\scaleint#1{\vcenter{\hbox{\scaleto[3ex]{\displaystyle\int}{#1}}}}
\usepackage{array}
\newcolumntype{C}{>{{}}c<{{}}} % for '+' and '-' symbols
\newcolumntype{R}{>{\displaystyle}r} % automatic display-style math mode 
\usepackage{tabularray}
\usepackage{dcolumn,tabularx,booktabs}
\usepackage{esvect}

%maths
\usepackage{mathtools}
\usepackage{amsmath}
\usepackage{amssymb}
\usepackage{amsfonts}
\usepackage{autobreak}

%tikzpicture
\usepackage{tikz}
\usepackage{scalerel}
\usepackage{pict2e}
\usepackage{tkz-euclide}
\usepackage{tikz-3dplot}
\usetikzlibrary{calc}
\usetikzlibrary{patterns,arrows.meta}
\usetikzlibrary{shadows}
\usetikzlibrary{external}
\usetikzlibrary{decorations.pathreplacing,angles,quotes}
\usetikzlibrary{perspective,spath3}

%pgfplots
\usepackage{pgfplots}
\pgfplotsset{compat=1.18}
\usepgfplotslibrary{statistics}
\usepgfplotslibrary{fillbetween}

\pgfplotsset{
    standard/.style={
    axis line style = thick,
    trig format=rad,
    enlargelimits,
    axis x line=middle,
    axis y line=middle,
    enlarge x limits=0.15,
    enlarge y limits=0.15,
    every axis x label/.style={at={(current axis.right of origin)},anchor=north west},
    every axis y label/.style={at={(current axis.above origin)},anchor=south east}
    }
}

\begin{document}
\phantom{c}
\vspace{70pt}
%\tdplotsetmaincoords{90}{90} % 90:0, 90:90, 0:90
\tdplotsetmaincoords{55}{60}
%\tdplotsetmaincoords{85}{10}
\begin{center}
\begin{tikzpicture}[tdplot_main_coords, scale=1]
\draw[rotate=30] (-4,-4,0) -- (-4,2.2,0) -- (7,2.2,0) -- (7,-4,0) -- cycle;
\node at (0.8,-5,0) {$\overline{\mathbb{C}}$};
\draw[-latex] (0,0,-3) -- (0,0,2); % z-axis
\draw[tdplot_screen_coords] (0,0) circle [radius=1]; % outer circle
\tdplotsetrotatedcoords{0}{0}{0}
\draw[tdplot_rotated_coords] (0,0) circle [radius=1]; % base intersection
\tdplotsetrotatedcoords{0}{21.8}{0}
\path[tdplot_rotated_coords,spath/save=PcircS] (0,0) circle [radius=0.3714];
\path[tdplot_rotated_coords,spath/save=PcircL] (0,0) circle [radius=0.3714];
\path[fill,opacity=0.5,spath/use={PcircS, transform={shift={(1*0.35582,1*0,1*0.862)}}}]; % NOTE: we swapped x and y for simplicity
\path[draw,spath/use={PcircS, transform={shift={(1*0.35582,1*0,1*0.862)}}}]; % NOTE: we swapped x and y for simplicity
\draw[] (2.5,-3) -- (2.5,2);
\draw[very thin] (0,0,1) -- (2.5,-2.2,0);
\draw[very thin] (0,0,1) -- (2.5,1.5,0);
\draw[very thin] (0,0,1) -- (2.5,0,0);
\node[left] at (0,0,1) {$\scriptscriptstyle N$};

\end{tikzpicture}
\end{center}

\vspace{70pt}
%\tdplotsetmaincoords{90}{90} % 90:0, 90:90, 0:90
\tdplotsetmaincoords{55}{10}
%\tdplotsetmaincoords{85}{10}
\begin{center}
\begin{tikzpicture}[tdplot_main_coords, scale=1]
\draw[] (-2,-4,0) -- (-2,3,0) -- (6,3,0) -- (6,-4,0) -- cycle;
\node at (-1.6,-3.2,0) {$\overline{\mathbb{C}}$};
\draw[-latex] (0,0,-2) -- (0,0,2); % z-axis
%\draw[-latex] (-2,0,0) -- (7,0,0) node[pos=1,above right]{$y,\zeta$}; % y-axis (swapped with the x for simplicity)
\draw[tdplot_screen_coords] (0,0) circle [radius=1]; % outer circle
\tdplotsetrotatedcoords{0}{0}{0}
\draw[tdplot_rotated_coords] (0,0) circle [radius=1]; % base intersection
\tdplotsetrotatedcoords{0}{37.87}{0}
\path[tdplot_rotated_coords,spath/save=PcircS] (0,0) circle [radius=0.26];
\path[tdplot_rotated_coords,spath/save=PcircL] (0,0) circle [radius=0.26];
\path[fill,opacity=0.5,spath/use={PcircS, transform={shift={(1*0.5923,1*0,1*0.7615)}}}]; % NOTE: we swapped x and y for simplicity
\path[draw,spath/use={PcircL, transform={shift={(1*0.5923,1*0,1*0.7615)}}}]; % NOTE: we swapped x and y for simplicity
\draw[fill,opacity=0.5] (3.5,0) circle [radius=1.5];
\draw[] (3.5,0) circle [radius=1.5];
\draw[very thin] (0,0,1) -- (3.5,0,0);
\draw[very thin] (0,0,1) -- (3.5,-1.5,0);
\draw[very thin] (0,0,1) -- (3.5,1.5,0);
\draw[very thin] (0,0,1) -- (2,0,0);
\node[left] at (0,0,1) {$\scriptscriptstyle N$};
\draw[fill,opacity=0.5] (0,0) circle [radius=0.26];
\path[spath/save=PcircS] (0,0) circle [radius=0.4877];\
\path[fill,opacity=0.5,spath/use={PcircS, transform={shift={(1*0,1*0,1*-0.873)}}}]; % NOTE: we swapped x and y for simplicity
% intersection circle\
\draw[very thin,dashed] (0,0,1) -- (0.4877,0,-0.873);
\draw[very thin,dashed] (0,0,1) -- (-0.4877,0,-0.873);
\end{tikzpicture}
\end{center}
\end{document}
